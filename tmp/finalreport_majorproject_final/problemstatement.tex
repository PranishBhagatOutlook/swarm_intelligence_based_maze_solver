\subsection{Problem Statement}
Swarm bots is a promising technology in recent robotics. But the construction of these bots is
complicated and tedious. We plan to design two swarm bots for solving a maze efficiently and in the
shortest possible path. Using a single robot creates difficulty in analyzing the shortest path. Moreover,
when multiple bots have to reach the same destination, it is faster with swarm-intelligence
implementation rather than multiple bots solving the maze independently.\\
The efficiency of the maze solving is boosted by the communication between the bots. The idea behind
swarm intelligence is reflected when two or more bots communicate with each other. With the age old
implementation of using single robots to solve a task, the process would not only be time consuming
but also less effective. Using multiple communicating bots greatly reduces the workload required on a
single bot and at the same time makes the entire process much more effective. For a task as simple as
solving a maze, a single robot would take multiple wrong turns and reach dead ends until it finally
reaches the target. When more than one robot is allowed to solve the maze in an integrated fashion, one
robot can work on one task while the other can work on another. For example, one can detect the target
while the other can decode the shortest path to it. With such communicating entity in place, the
underlying complexity for each bot is greatly reduced.\\
Maze solving alone is a much studied topic in robotics. The real world field of applications
implementing maze solving is quite diverse. Scenarios such as emergency evacuation in disaster
stricken areas require autonomous devices to traverse several obstacles to reach the destination which
might or might not be known firsthand. Similarly, in hostage rescue situations where the lives of not
only the hostages but also the rescuers matter equally, it is highly advisable to use such robots.
Unknown path finding and localization also implement maze solving techniques and algorithms.
Due to the lack of methods and tools, swarm robot designers cannot achieve the complexity required
for the real world applications. This project is just the small simulation in real world. The developed
swarm robot approach uses decentralized control strategies within the swarm of heterogeneous robots.
The robot-to-robot and robot-to-environment interaction provides the task oriented, simple collective
swarm behavior. We have decided to control the robot based on differential kinematics rather than
forward and inverse kinematics. This is done to avoid slippage, collision and cost. DC motors will be
used rather than stepper motors because of the speed limitation in the later one. IR sensors will be used
to detect the obstacles in the maze where the information received by these sensors will be used to
calculate the shortest path to the destination. The destination will be initially unknown and a smoke
sensor will be implemented to detect the target where a source of smoke will be placed. Arduino
microcontroller will be the brain of the robot where bluetooth technology will be used to establish a
communication link between the two robots. Hence, the advantages and application of swarm
intelligence is aimed to be demonstrated through successful implementation of this project.