\section{CONCLUSION}
Thus, we were able to successfully meet the desired objectives of the project within the given time period. Some minor shortcomings were experienced and were successfully tackled.  
\subsection{Limitation} 
The objectives specified initially was completed in the given time and with the available resources with some minor changes along the development process. The project still has some limitations which are to be solved in future revisions. They are as listed below: 
\begin{enumerate}
\item The accuracy of the IR sensors was very less and produced erroneous results when directly subjected to sunlight, requiring a closed room to function properly.
\item Use of R sensors makes the robot incapable to trace the map of the maze that it traverses.
\item The bluetooth module HC-05 cannot be used for establishing connection between more than two robots. 
\item Use of wall follower algorithm to solve the maze limits the structure of the maze to be simple connected (i.e. all its walls are connected together or to the maze's outer boundary).
\end{enumerate}
\subsection{Future Enhancement}
The enhancements to be implemented in the project to overcome the limitations stated above as well as to provide advanced features are as follows:
\begin{enumerate}
\item Use of robust IR sensors that work under a variety of light conditions and be able to detect the distance from the obstacles.
\item Integration of a camera with the robots to map the maze surrounding and implement dynamic path planning.
\item Using IEEE 802.11 protocol in the system in order to implement more than two robots in the maze.
\item Implementing a stronger algorithm to make all the robots scan and traverse the maze at the same time simultaneously.
\end{enumerate}
\subsection{Recommendations}
This project is a demonstration of the concept of swarm robotics in task solving and can be applied to various applications. The project uses two robots to solve a maze and can be used in disaster zones where it is dangerous for humans to undertake the rescue operation like search and rescue operations (SAR). It can also be used in traversing an unknown environment where the target location is not determined such as in hostage rescue situations. By prevailing over the existing
limitations, it can be used for land-mine detection where one robot detects the land-mine and the other robot goes to the location to diffuse it.
