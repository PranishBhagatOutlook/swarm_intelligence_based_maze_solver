\subsection{Background}
In robotics, autonomous navigation is an important feature because it allows the robot to independently
move from a point to another without human intervention. Autonomous navigation within an unknown
area requires the robot to explore, localize and map its surrounding.\\
By solving a maze, the pertaining algorithms and behavior of the robot can be studied and
improved upon.\\
Swarm robotics is a new approach to the coordination of multirobot systems which consists
of large numbers of mostly simple physical robots. It is supposed that a desired collective
behavior emerges from the interactions between the robots and interactions of robots with
the environment. This approach emerged on the field of artificial swarm intelligence, as well
as the biological studies of insects, ants and other fields in nature, where swarm behavior
occurs.\\
Maze-solving is a structured technique and controlled implementation of autonomous navigation which
is sometimes preferable in studying specific aspect of the problem. These
robots are also known as “micro-mouse” robots. Research on various algorithms to efficiently
solve complex mazes has been ongoing since the past 50 years.\\
One of the most amazing developments in biological evolution is the domain of social insects. These
animals, although being very small, achieve impressive feats. Bees, ants and
termites live in elaborately constructed nests which are, in comparison to the insect, gigantic. The
colony super-organism can be characterized as being swarm-intelligent because its
abilities, for example to optimally allocate foragers to food sources, are a result of the interactions
within the swarm and cannot be achieved by the single individual. This decentralized
and distributed way of achieving a goal is an interesting and useful field of study which has
inspired the fields of swarm-intelligence and swarm robotics. In the last decade, a lot of
control strategies and algorithms for robotic swarms have been presented, both in simulated
and real robot swarms.\\
Maze-solving can be achieved using various sensing methods. For the robot to traverse the unknown
surrounding and identify obstacles successfully, it needs the help of certain sensors. IR sensors and
ultrosonic sensors have been widely used in such robots. Among them, IR sensors are more widely
used because of their narrow range of field whereas ultrasonic sensors have a wider sensing area. Light
Detection and Ranging (LiDAR) is another technology that has seen recent developments in use for
autonomous navigation. It is a surveying technology that measures distance by illuminating a target
with a laser light.\\
Bluetooth is a wireless technology standard for exchanging data over short distances (using
short-wavelength UHF radio waves in the ISM band from 2.4 to 2.485 GHz) from fixed and
mobile devices, and building personal area networks (PANs). Invented by telecom vendor
Ericsson in 1994,it was originally conceived as a wireless alternative to RS-232 data cables.
It can connect several devices, overcoming problems of synchronization but the bluetooth
module namely HC-05 used in our project is only able to connect to a single device i.e. a
single HC-05 is able to connect to another HC-05 or a personal computer wirelessly.
Bluetooth is managed by the Bluetooth Special Interest Group (SIG), which has more than
25,000 member companies in the areas of telecommunication, computing, networking, and
consumer electronics. The IEEE standardized Bluetooth as IEEE 802.15.1, but no longer
maintains the standard. The Bluetooth SIG oversees development of the specification, manages the qualification program, and protects the trademarks. A manufacturer must make a
device meet Bluetooth SIG standards to market it as a Bluetooth device.