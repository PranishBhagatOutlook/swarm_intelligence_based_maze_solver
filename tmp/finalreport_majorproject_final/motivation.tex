\subsection{Motivation}
The idea behind our final year project came from the concept of ‘Swarm Robotics’. This
is a branch of robotics which is still in its infancy. Swarm robotics is the study of coordination of large
groups of relatively simple robots through the use of local rules. It takes its
inspiration from societies of insects that can perform tasks that are beyond the capabilities of
the individuals. Beni\cite{beni}, scholar, University of California describes this kind of robots’ coordination as follows:
”The group of robots is not just a group. It has some special characteristics, which are
found in swarms of insects, that is, decentralized control, lack of synchronization, simple
and (quasi) identical members.”\\
We want to demonstrate this concept using two robots which navigate their way through a
maze. The first bot will solve the maze individually taking as much time as it requires to reach an
unknown destination. The second one after communication with the first bot will solve the same maze
in the shortest path possible. While our project implements just two robots, any further robot that is
added to the system will solve the maze in the shortest path as well.\\
Our motivation arises from the behavior of insects such as ants which behave in a coordinated manner
and coordinating a group of robots in similar ways can ultimately make the
completion of any task simpler and less time consuming. Aside from this, we wanted to
implement a simple maze solving algorithm which can be implemented with minimal hardware
requirements keeping in mind the constraints in availability of materials in the local market.
Among the various methods to establish a communication link between the two robots, bluetooth
technology was chosen for its simplicity and ease of use within a short range.\\
Furthermore, in order to further justify the concept of swarm robotics we wanted to implement a simple
sensing circuitry. IR sensors were used as a result. With the use simple sensors and a simple algorithm
to traverse the maze and find the shortest path, we wanted to lay a foundation on the idea of swarm
intelligence upon which further additions could be done.\\
An advanced and real world use of our project which fueled our motivation to strive forward
with this idea was to implement this concept in Search and Rescue (SAR) operation, Land-mine
Detection, Hostage Rescue situations, etc. With slight modifications, mainly to the
mechanical parts and components, our project can be extended to a much larger and complex scale.
This turned out to be one of the most prominent sources of our motivation.