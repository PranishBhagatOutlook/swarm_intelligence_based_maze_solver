\section{THEORETICAL BACKGROUND}
\subsection{Swarm Robotics}
Swarm robotics is the study of how to design groups of robots that operate without relying on any external infrastructure or on any form of centralized control. In a robot swarm, the collective behavior of the robots results from local interactions between the robots and between the robots and the environment in which they act. The design of robot swarms is guided by swarm intelligence principles. These principles promote the realization of systems that are fault tolerant, scalable and flexible. Swarm robotics appears to be a promising approach when different activities must be performed concurrently, when high redundancy and the lack of a single point of failure are desired, and when it is technically infeasible to set up the infrastructure required to control the robots in a centralized way. Examples of tasks that could be profitably tackled using swarm robotics are demining, search and rescue, planetary or underwater exploration, and surveillance.\\
Swarm robotics has its origins in swarm intelligence and, in fact, could be defined as "embodied swarm intelligence". Initially, the main focus of swarm robotics research was to study and validate biological research. Early collaboration between roboticists and biologists helped bootstrap swarm robotics research, which has since become a research field in its own right. In recent years, the focus of swarm robotics has been shifting: from a bio-inspired field of robotics, swarm robotics is increasingly becoming an engineering field whose focus is on the development of tools and methods to solve real problems.\\
A robot swarm is a self-organizing multi-robot system characterized by high redundancy. Robots’ sensing and communication capabilities are local and robots do not have access to global information. The collective behavior of the robot swarm emerges from the interactions of each individual robot with its peers and with the environment. Typically, a robot swarm is composed of homogeneous robots, although some examples of heterogeneous robot swarms do exist.\\ 
The research of swarm robotics is to study the design of robots, their physical body and their controlling behaviors. It is inspired but not limited by the emergent behavior observed in social insects, called swarm intelligence. Relatively simple individual rules can produce a large set of complex swarm behaviors. A key-component is the communication between the members of the group that build a system of constant feedback. The swarm behavior involves constant change of individuals in cooperation with others, as well as the behavior of the whole group.\\
The aforementioned characteristics of swarm robotics are deemed to promote the realization of systems that are fault tolerant, scalable and flexible.\\
Swarm robotics promotes the development of systems that are able to cope well with the failure of one or more of their constituent robots: the loss of individual robots does not imply the failure of the whole swarm. Fault tolerance is enabled by the high redundancy of the swarm: the swarm does not rely on any centralized control entity, leaders.\\
Swarm robotics also enables the development of systems that are able to cope well with changes in their group size: ideally, the introduction or removal of individuals does not cause a drastic change in the performance of the swarm. Scalability is enabled by local sensing and communication: provided that the introduction and removal of robots does not dramatically modify the density of the swarm, each individual robot will keep interacting with approximately the same number of peers, those that are in its sensing and communication range.\\
Finally, swarm robotics promotes the development of systems that are able to deal with a broad spectrum of environments and operating conditions. Flexibility is enabled by the distributed and self-organized nature of a robot swarm: in a swarm, robots dynamically allocate themselves to different tasks to match the requirements of the specific environment and operating conditions; moreover, robots operate on the basis of local sensing and communication and do not rely on pre-existing infrastructure or on any form of global information.
\subsubsection{Design of Swarm Robots}
The design of a robot swarm is a difficult endeavor: requirements are usually expressed at the collective level, but the designer needs to define hardware and behavior at the level of individual robots. The resulting robots should interact in such a way that the global behaviour of the swarm meets the desired requirements. Approaches to the design problem in swarm robotics can be divided into two categories: manual design and automatic design.\cite{swarmrobot}\\
In manual design, the designer follows a trial-and-error process in which the behaviors of the individual robot are developed, tested and improved until the desired collective behavior is obtained. The software architecture that is most commonly adopted in swarm robotics is the probabilistic finite state machine. Probabilistic finite state machines have been used to obtain several collective behaviors.
\subsubsection{Swarm Intelligence}
As an emerging research area, the swarm intelligence has attracted many researchers' attention since the concept was proposed in 1980s. It has now become an interdisciplinary frontier and focus of many disciplines including artificial intelligence, economics, sociology, biology, etc. It has been observed a long time ago that some species survive in the cruel nature taking the advantage of the power of swarms, rather than the wisdom of individuals. The individuals in such swarm are not highly intelligent, yet they complete the complex tasks through cooperation and division of labor and show high intelligence as a whole swarm which is highly self-organized and self-adaptive.\\
Swarm intelligence is the discipline that deals with natural and artificial systems composed of many individuals that coordinate using decentralized control and self-organization. In particular, the discipline focuses on the collective behaviors that result from the local interactions of the individuals with each other and with their environment. Examples of systems studied by swarm intelligence are colonies of ants and termites, schools of fish, flocks of birds, herds of land animals. Some human artifacts also fall into the domain of swarm intelligence, notably some multi-robot systems, and also certain computer programs that are written to tackle optimization and data analysis problems.\\
Swarm intelligence is a soft bionic\cite{swarmintel} of the nature swarms, i.e. it simulates the social structures and interactions of the swarm rather than the structure of an individual in traditional artificial intelligence. The individuals can be regarded as agents with simple and single abilities. Some of them have the ability to evolve themselves when dealing with certain problems to make better compatibility. A swarm intelligence system usually consists of a group of simple individuals autonomously controlled by a plain set of rules and local interactions. These individuals are not necessarily unwise, but are relatively simple compared to the global intelligence achieved through the system. Some intelligent behaviors never observed in a single individual will soon emerge when several individuals begin cooperate or compete. The swarm can complete  the tasks that a complex individual can do while having high robustness and flexibility and low cost. Swarm intelligence takes the full advantage of the swarm without the need of centralized control and global model, and provides a great solution for large-scale sophisticated problems.\\
The typical swarm intelligence system has the following properties:
\begin{enumerate}
\item It is composed of many individuals.
\item The individuals are relatively homogeneous.
\item The interactions among the individuals are based on simple behavioral rules that exploit only local information that the individuals exchange directly or via the environment.
\item The overall behaviour of the system results from the interactions of individuals with each other and with their environment, that is, the group behavior self-organizes.
\end{enumerate}
The characterizing property of a swarm intelligence system is its ability to act in a coordinated way without the presence of a coordinator or of an external controller. Many examples can be observed in nature of swarms that perform some collective behavior without any individual controlling the group, or being aware of the overall group behavior. Notwithstanding the lack of individuals in charge of the group, the swarm as a whole can show an intelligent behavior. This is the result of the interaction of spatially neighboring individuals that act on the basis of simple rules.
\subsubsection{Advantages of Swarm Robotics}
The advantages and characteristics of the swarm robotics system are presented by comparing a single robot and other similar systems with multiple individuals.
\subsubsection*{Comparison with a Single Robot}
To complete a sophisticated task, a single robot must be designed with complicated structure and control modules resulting in high cost of design, construction and maintenance. Single robot is vulnerable especially when a small broken part of the robot may affect the whole system and it's difficult to predict what will happen. The swarm robotics can achieve the same ability through inter-group cooperation and takes the advantage of reusability of the simple agents and the low cost of construction and maintenance. The swarm robotics also takes the advantage of high parallelism and is especially suitable for large scale tasks.\\
A single robot is inspired from human behaviors by comparing the corresponding nature species of these researching areas, while the swarm robotics is inspired from the social animals. Due to the restriction of current technology, it's hard to simulate the human interactions using machines or computers while the mechanisms in animal groups are easier to apply. This gives the swarm robotics a bright future in dealing with complex and large scale problems. The advantages of swarm robotics are described below.
\begin{enumerate}
\item Exploit Parallelism: 
The population size of swarm robotics is usually quite large, and it can deal with multiple targets in one task. This indicates that the swarm can perform the tasks involving multiple targets distributed in a vast range in the environment, and the search of the swarm would save time significantly.
\item Exploit Scalability:
The interaction in the swarm is local, allowing the individuals to join or quit the task at any time without interrupting the whole swarm. The swarm can adapt to the change in population through implicit task re-allocating schemes without the need of any external operation. This also indicates that the system is adaptable for different sizes of population without any modification of the software or hardware which is very useful for real-life application.
\item Exploit Stablility :
Similar to scalability, the swarm robotics systems are not affected greatly even when part of the swarm quits due to the major factors. The swarm can still work towards the objective of the task although their performances may degrade inevitably with fewer robots. This feature is especially useful for the tasks in a dangerous environment.
\subsubsection{Open Issues/Demerits}
Despite its potential to promote robustness, scalability and flexibility, swarm robotics has yet to be adopted for solving real-world problems. Various limiting factors are preventing the real-world uptake of swarm robotics systems. Further research is needed on robotic hardware to overcome hardware shortcomings that limit the functionality of current robotic systems, while further research on behavioural control is needed to discover effective ways to let a human operator interact with a robot swarm. More effort is required to provide compelling case-studies—in particular to demonstrate swarm robotics in outdoor applications (e.g., waste removal), but also to develop business cases and business models that show how and where swarm robotics can be more effective than other approaches. Finally, an engineering methodology is still lacking for swarm robotics systems, which would include the definition of standard metrics, performance assessment testbeds and formal analysis techniques to verify and guarantee the properties of swarm robotics systems.
\end{enumerate}