\section{INTRODUCTION}
The purpose of this project is to design a multi-robot system which solves a simple maze using short range bluetooth communication. The project aims to create a robust communication system between two robots that would allow for the transmission of maze information from one maze-traversing bot to the other. \\
Although wireless communication protocols are relatively commonplace in the field of robotics, the use of this technology for coordinated robotics is somewhat different. Thus, the main contribution of this project is to outline the details of using wireless communication for coordinated swarm robotics.
Swarm robotics offers an intriguing approach to many real-world engineering problems. Such tasks that require covering a large search space or that involve working in potentially hazardous environments naturally lend themselves to robotic and swarm solutions. In many cases, autonomous robots can be the best option for a specific mission. Unmanned autonomous robotic vehicles which can enter the collapsed builds searching for survivors maybe a solution of finding survivors faster and safer. Being equipped with the necessary sensory devices, unmanned autonomous robotic vehicles can scan the environment sending precious information to the rescue teams about the location of survivors.\\
We have developed a multi-robot system designed to use short range wireless communication (Bluetooth) to solve a simple maze. A Maze Solving robot also known as a 'Micro-Mouse' robot is an autonomous vehicle which uses simple sensors to solve a maze without any human intervention. Maze-solving, although artificial in terms of the confine that the robot it subjected to, is a structured technique and controlled implementation of autonomous navigation. The choice of algorithms was limited to ones that required direct bot-to-bot communication and using bots with a limited sensing range. The Left Wall Follower Algorithm was implemented on the first bot, in the process showing the advantages and weakness of this technique. The bot will use data from simple obstacle-avoiding IR sensors to traverse through the maze whilst following the left wall. A maze, the complexity of which can be changed by moving its walls within the maze boundary will be used to test the system. Unlike in 'Micro-Mouse' systems, the destination of the maze will be unknown to the bot. Using suitable sensors, the first bot will reach the destination upon which it will transfer the maze information to the second bot. Hence, the second bot will traverse the maze based on the information provided by the first bot. 
Based on the robot reaching the target at the beginning, the shortest path to the target/destination will be calculated. The second bot will traverse the maze in the hence calculated shortest path.\\
Using the Arduino Uno chipset to program both the bots, the concept of swarm  robotics is well established in the project. The communication between two bots, one having higher complexity makes it easier for the other bot to be implemented with a lower complexity grade and still make it more intelligent. This concept can be well visualized in areas of emergence where a single vehicle cannot fulfill the task as effectively as a swarm of communicating bots can. The second bot and any other further bots that may/can be implemented in the system will inherently require less hardware and will always solve the maze in the shortest and most optimum path. This is the gist of swarm robotics. 
The concept of maze solving can be implemented in various real life situations as well. Natural disaster zones, hostage rescue situations, navigation in unknown territory are some examples. The left wall algorithm is one of the simplest and efficient methods to solve a maze. While moving through a maze it always follows the left wall, in meaning that at a junction it gives the most priority to the left turn. Future implementations of this algorithms might include a hybrid wall follower algorithm that implements both the left and right wall algorithms intelligently. Wireless communication using bluetooth technology allows the two bots to be completely devoid of human involvement during it's time of operation. Establishment of a communication pathway between two Arduinos using a bluetooth module is considered to be a significant milestone for the project. \\
Further topics following this will present the project and its method in a much greater detail and provide better insight into the technology implemented to achieve the project goal.

