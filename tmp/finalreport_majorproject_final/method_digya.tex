\subsection{Locomotion of the Robot}
\subsubsection{Differential Kinematics}
\justify
The robot consists of two drive wheels mounted on a common axis, that is, it uses differential kinematics. Each wheel can be independently driven either forward or backward. While we vary the velocity of each wheel, for the robot to perform rolling motion, the robot rotate about a point that lies along their common left and right wheel axis. The point that the robot rotates about is known as the ICC - Instantaneous Center of Curvature.\\
\begin{figure}[h]
\center
\includegraphics[scale=0.75]{differential2.png}    
\caption{Arrangement for Differential Drive Kinematics}
\end{figure}
\justify 
By varying the velocities of the two wheels, we can vary the trajectories that the robot takes.
Because the rate of rotation $\omega$ about the ICC must be the same for both wheels, we can write the
following equations:\\
\begin{equation}
                                                                         		\omega(R+l/2)= Vr                                             
\end{equation}

 \begin{equation}
   \omega(R-l/2)= Vl   
 \end{equation}       
\justify
where l is the distance between the centers of the two wheels, Vr ; Vl are the right and left wheel
velocities along the ground , and R is the signed distance from the ICC to the midpoint between the wheels. At any instance in time we can solve for R and $\omega$:
\begin{equation}
                                                                         		                                                        R= l/2 (Vl+ Vr)/(Vl-  Vr)                                                  
\end{equation}
\begin{equation}
                                                                      	                                                                 \omega= ( Vr-Vl)/l                                                    	                                                      \end{equation}
\justify
There are three interesting cases with these kinds of drives.\\
\justify
\begin{enumerate}
\item If Vl = Vr, then we have forward linear motion in a straight line. R becomes infinite, and there
is effectively no rotation - $\omega$ is zero.
\item If Vl = -Vr, then R = 0, and we have rotation about the midpoint of the wheel axis, we rotate
in place.
\item If Vl = 0, then we have rotation about the left wheel. In this case R = l/2. Same is true if Vr = 0.
\end{enumerate}
\subsubsection*{Operation of H-Bridge:}
The basic operating mode of an H-bridge is fairly simple: if Q1 and Q4 are turned on, the left lead of the motor will be connected to the power supply, while the right lead is connected to ground. Current starts flowing through the motor which energizes the motor in (let’s say) the forward direction and the motor shaft starts spinning.
\begin{figure}[h]
\center
\includegraphics[scale=0.6]{photos/13281783_10154154713090270_1743784099_n.png} 
\caption{Current flowing in the forward direction}
\end{figure}
\justify
If Q2 and Q3 are turned on, the reverse will happen, the motor gets energized in the reverse direction, and the shaft will start spinning backwards.\\
\newpage
\begin{figure}[h]
\center
\includegraphics[scale=0.6]{photos/13285612_10154154713070270_1233792274_n.png} 
\caption{Current flowing in the backward direction}
\end{figure}
\justify
In a bridge, both Q1 and Q2 (or Q3 and Q4) should never be closed at the same time. If done so, a really low-resistance path between power and GND is created effectively short-circuiting power supply. This condition is called ‘shoot-through’ and is an almost guaranteed way to quickly destroy the bridge, or something else in the circuit.\\

\begin{figure}[h]
\center
\includegraphics[scale=1]{photos/13288876_10154154713075270_284275638_n.png} 
\caption{Occurrence of short-circuit}
\end{figure}
\justify
The H-bridge arrangement is generally used to reverse the polarity/direction of the motor, but can also be used to 'brake' the motor, where the motor comes to a sudden stop, as the motor's terminals are shorted, or to let the motor 'free run' to a stop, as the motor is effectively disconnected from the circuit. The following table summarizes operation, with Q1-Q4 corresponding to the diagrams above.\\
\begin{table}[h]
\begin{center}
\begin{tabular}{ |c|c|c|c|c| } 
 \hline
 Q1 & 	Q2 & Q3&Q4&RESULT\\ 
 \hline
 1 & 0 & 0 &1& Motor moves right \\ 
 \hline
 0 &  1&  1&0& Motor moves left \\ 
\hline
 0 & 0 & 0 & 0 & Motor coasts \\
\hline
 0&1&0&1&Motor brakes  \\
\hline
1&0&1&0&Motor brakes \\
\hline
1&1&0&0& Short Circuit\\
\hline
0&0&1&1& Short Circuit \\
\hline
1 &1&1&1& Short Circuit\\
 \hline
\end{tabular}
\caption{Operation of an H-bridge}
\end{center}
\end{table}
\newpage
\subsubsection*{PWM and H-Bridge for the operation of DC motor}
For the generation of PWM, analogWrite and digitalWrite functions are used in Arduino.
analogWrite controls speed and digitalWrite controls direction of DC motor.
DC motors normally have just two leads, one positive and one negative. If these two leads are directly connected to a battery, the motor will rotate. If you switch the leads, the motor will rotate in the opposite direction.\\
To control the direction of the spin of DC motor, without changing the way that the leads are connected, a circuit called an H-Bridge can be used. \\
We have used L298N H-Bridge IC. The L298N can control the speed and direction of DC motors and stepper motors and can control two motors simultaneously. Its current rating is 2A for each motor. So, heat sinks are used.\\
\newpage
